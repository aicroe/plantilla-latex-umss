\documentclass[
  12pt,                       % Tamaño de letra, 12 puntos (ejemplos: 10pt,11pt,12pt)
  letterpapper,               % Tamaño de papel. Puede ser 'a4paper' (a4), letterpaper (carta), otros
  twoside,                    % Se dispondrán las hojas para generar un archivo imprimible en amberso y reverso. Usar 'oneside' ṕara indicar lo contrario
  openright                   % Un capítulo siempre iniciará en una hoja de la derecha. Usar 'openany' para iniciar un capítulo en cualquier hoja
]{book}                       % Tipo de documento, 'book' es adecuado para documentos de trabajo de grado o tesis. Otras opciones disponibles son 'report' y 'article'

% Preámbulo

%% Importación de paquetes
%%% Paquetes esenciales

\usepackage[utf8]{inputenc}   % Aceptar entrada de texto con codificación utf-8. https://ctan.org/pkg/inputenc
\usepackage{hyperref}         % Paquete para manejar links de hipertexto en el documento. https://ctan.org/pkg/hyperref
\usepackage{apacite}          % Utilizar APA para las citar la bibliografía. Las citas se manejan de forma similar a links de hipertexto. Importar después de hyperref. https://ctan.org/pkg/apacite
\usepackage{appendix}         % Paquete que permite personalizar los anexos. Puede tener problemas con dependencias que manejan traducciones, importar antes que babel. https://ctan.org/pkg/appendix
\usepackage[                  % Soporte para diferentes idiomas
  spanish,                    % Seleccionar español como idioma objetivo para traducir las etiquetas
  es-tabla                    % Importar soporte de español para tablas
]{babel}                      % Este paquete modifica comandos de citas bibliográficas importar después de 'apacite'. https://ctan.org/pkg/babel
\usepackage{mathptmx}         % Selecciona Times Roman como la fuente por defecto para generar el imprimible, y provee soporte básico para símbolos matemáticos. https://ctan.org/pkg/mathptmx
\usepackage{paralist}         % Provee los ambientes para insertar listas numeradas y no numeradas. https://ctan.org/pkg/paralist
\usepackage{fancyhdr}         % Provee facilidades para construir cabeceras y pies de página. https://ctan.org/pkg/fancyhdr
\usepackage{geometry}         % Importa interfaz para personalizar los márgenes de página. https://ctan.org/pkg/geometry
\usepackage{titlesec}         % Importa interfaz para personalizar el estilo de los títulos. https://ctan.org/pkg/titlesec
\usepackage{graphicx}         % Provee comandos para añadir imágenes al documento. https://ctan.org/pkg/graphicx
\usepackage{setspace}         % Provee comandos para configurar el espacio entre líneas de un párrafo. https://ctan.org/pkg/setspace
\usepackage{emptypage}        % Evita que los números de página o cabeceras aparezcan en hojas vacias. https://ctan.org/pkg/emptypage
\usepackage{chngcntr}         % Cambia la forma de enumerar una etiqueta (Por ejemplo: tablas o figuras). Define los comandos '\counterwithout' y '\coun­ter­within'. https://ctan.org/pkg/chngcntr

%%% Paquetes que pueden ser añadidos según la necesidad

%\usepackage{makecell}        % Permite tener celdas multilínea en tablas. https://ctan.org/pkg/makecell
%\usepackage{amsmath}         % Importa símbolos que pueden ser usados dentro el modo matemático. https://ctan.org/pkg/amsmath
%\usepackage[                 % Importa el ambiente 'algorithm' para definir algoritmos en cuadros flotantes
%  ruled,                     % Apariencia del typeset del algoritmo. Puede ser: 'plain', 'boxed', 'ruled'
%  nothing                    % Nivel al que se realizará la numeración de los algoritmos en el documento. Puede ser: 'part', 'chapter', 'section', 'subsection', 'subsubsection', 'nothing'
%]{algorithm}                 % https://ctan.org/pkg/algorithms
%\usepackage{algorithmic}     % Hace público comandos de algorithms con tal de personalizar el algoritmo generado

%% Configuración del documento

\graphicspath{                % Define rutas de donde se pueden obtener las imágenes
  {./Capitulo1/}              % Se añaden las carpetas de todos los capítulos como rutas que pueden contener imágenes
  {./Capitulo2/}
  {./Capitulo3/}
  {./Capitulo4/}
  {./Capitulo5/}
  {./Anexos/}
}

\geometry{
  letterpaper,                % Tamaño del papel
  left=3cm,                   % Margen izquierdo
  right=2.5cm,                % Margen derecho
  top=2.5cm,                  % Margen superior
  bottom=2.5cm                % Margen inferior
}

\pagestyle{plain}             % Hace que el documento generado este numerado. El valor 'empty' evita que se enumeren las páginas

% La numeración de tablas y figuras se realiza por capítulos por defecto.
% Si se quiere enumerarlos de forma absoluta, es decir sin reiniciar la numeración cada capítulo, descomenta las siguientes dos líneas
%\counterwithout{figure}{chapter}
%\counterwithout{table}{chapter}

% Las siguientes 4 líneas logran evitar que las palabras se corten automáticamente con un guión al final de una línea
%\tolerance=1
%\emergencystretch=\maxdimen
%\hyphenpenalty=10000
%\hbadness=10000

%%% Definir un nuevo ambiente donde irá el resumen del documento
\newenvironment{abstract}
  {
    \null\vfill
    \begin{center}
      \bfseries\MakeUppercase\abstractname
    \end{center}
  }
  {
    \vfill\null
  }

%%% Modificar los títulos de cada capítulo
\titleformat{\chapter}
  [hang] 
  {\normalfont\LARGE\bfseries\MakeUppercase} % Rutina que modifica el título del capítulo. 
  {\chaptertitlename\ \thechapter}           % El título del capítulo
  {1em}                                      % Separación horizontal entre la etiqueta y el título del capítulo
  {}

%%% A veces es necesario renombrar las etiquetas usadas por LaTeX

%%%% La siguiente línea renombra el título de la bibliografía que es traducido por babel
\addto{\captionsspanish}{\renewcommand{\bibname}{BIBLIOGRAFÍA}}
%\addto{\captionsspanish}{\renewcommand{\figurename}{Imágen}}     % Si necesitará cambiar la etiqueta de 'Figura' des-comente esta línea

%%%% Las siguientes líneas renombran el título de los anexos de la tabla de contenidos y su sección
\renewcommand{\appendixpagename}{ANEXOS}
\renewcommand{\appendixtocname}{ANEXOS}

% Inicio del documento
\begin{document}

%% Caratula
\begin{titlepage}

  %% Reemplazar los valores que se encuentran entre corchetes
  \thispagestyle{empty}
  \parbox[c][.125\textheight]{.15\textwidth}{
    \includegraphics[height=.125\textheight]{umss-logo}
  }
  \hfill
  \parbox[c]{.5\textwidth}{
    \small
    UNIVERSIDAD MAYOR DE SAN SIMÓN \\
    FACULTAD DE CIENCIAS Y TECNOLOGÍA \\
    CARRERA [[NOMBRE DE LA CARRERA]]
  }
  \hfill
  \parbox[c]{.15\textwidth}{
    \includegraphics[width=.15\textwidth]{fcyt-logo}
  }

  \vspace{.1\textheight}
  \begin{center}
    \parbox[c]{.8\textwidth}{
      \centering\Large
      \textbf{[[Plantilla de \LaTeX\ para la facultad de ciencias y tecnología de la umss]]}
      \par
    }
  \end{center}

  \vspace{.1\textheight}

  \begin{center}
    \Large
    [[Modalidad de Titulación]], Presentada Para Obtar al Diplóma Académico de [[Nombre de la Carrera]].
  \end{center}

  \vspace{12pt}

  \begin{center}
    \parbox[c]{.8\textwidth}{
      \large
      \textbf{Presentado por:} [[Nombre Estudiante]] \\
      \textbf{Tutor:} [[Nombre Tutor]] \\
    }
  \end{center}

  \null\vspace{.1\textheight}

  {\centering\Large
    [[Departamento]] - Bolivia \\
    ((Mes)), ((Año))
    \par
  }
\end{titlepage}

\frontmatter    % Define la sección que está antes de la tabla de contenido. Numerado con números romanos

%% Dedicatoria
\addcontentsline{toc}{chapter}{DEDICATORIA}
\begin{flushright}
    \null\vspace{\stretch{1}}
        \textit{\textbf{Dedicatoria}} \\
        \input{Dedicatoria.tex}
    \vspace{\stretch{2}}\null
\end{flushright}
\cleardoublepage

%% Agradecimientos
\addcontentsline{toc}{chapter}{AGRADECIMIENTOS}
\begin{flushright}
    \null\vspace{\stretch{1}}
        \textit{\textbf{Agradecimientos}} \\
        \input{Agradecimientos.tex}
    \vspace{\stretch{2}}\null
\end{flushright}
\cleardoublepage

%% Resumen
\onehalfspace
\addcontentsline{toc}{chapter}{RESUMEN}
\begin{abstract}
    \input{Resumen}
\end{abstract}

%% Tabla de contenido
\singlespace    % Interlineado 1.0 para las tablas de contenidos
\tableofcontents
\listoffigures
\listoftables

\mainmatter     % Define la sección que contiene el contenido principal del documento. Numerado normal
\onehalfspace   % Interlineado 1.5 para el resto del documento
\setlength{\parskip}{6pt}

% Importar los capítulos de sus carpetas. Asegúrese de no utilizar caracteres como espacios, ñ, ó (los guiones si son aceptados) cuando nombre los archivos
\chapter{INTRODUCCIÓN} \label{cap:1}          % Anexar una etiqueta al título del capítulo hace sencillo hacer referencias a él más adelante en el documento

Lorem ipsum dolor sit amet, consectetur adipiscing elit, sed eiusmod tempor incidunt ut labore et dolore magna aliqua. Ut enim ad minim veniam, quis nostrud exercitation ullamco laboris nisi ut aliquid ex ea commodi consequat. Quis aute iure reprehenderit in voluptate velit esse cillum dolore eu fugiat nulla pariatur. Excepteur sint obcaecat cupiditat non proident, sunt in culpa qui officia deserunt mollit anim id est laborum.

\section{Objetivos}

Lorem ipsum dolor sit amet, consectetur adipiscing elit, sed eiusmod tempor incidunt ut labore et dolore magna aliqua. Ut enim ad minim veniam, quis nostrud exercitation ullamco laboris nisi ut aliquid ex ea commodi consequat. Quis aute iure reprehenderit in voluptate velit esse cillum dolore eu fugiat nulla pariatur. Excepteur sint obcaecat cupiditat non proident, sunt in culpa qui officia deserunt mollit anim id est laborum.

\section{Estado del arte}

Lorem ipsum dolor sit amet, consectetur adipiscing elit, sed eiusmod tempor incidunt ut labore et dolore magna aliqua. Ut enim ad minim veniam, quis nostrud exercitation ullamco laboris nisi ut aliquid ex ea commodi consequat. Quis aute iure reprehenderit in voluptate velit esse cillum dolore eu fugiat nulla pariatur. Excepteur sint obcaecat cupiditat non proident, sunt in culpa qui officia deserunt mollit anim id est laborum \cite{bib01}.

\section{Motivación}

Lorem ipsum dolor sit amet, consectetur adipiscing elit, sed eiusmod tempor incidunt ut labore et dolore magna aliqua. Ut enim ad minim veniam, quis nostrud exercitation ullamco laboris nisi ut aliquid ex ea commodi consequat. Quis aute iure reprehenderit in voluptate velit esse cillum dolore eu fugiat nulla pariatur. Excepteur sint obcaecat cupiditat non proident, sunt in culpa qui officia deserunt mollit anim id est laborum \cite{bib01}.

\subsection{Insertar figura}

\begin{figure}[h]                             % Insertar aquí (h=here). Latex tomará en cuenta esta configuración pero si no es posible insertarla aquí, latex buscará el mejor lugar
  \centering                                  % Centrado
  \includegraphics[scale=0.5]{cuadrado-azul}  % Insertar el archivo 'cuadrado-azul(.png/.jpg/...)' escalado al 50%
  \caption[Cuadrado azul]{Cuadrado azul (Elaboración propia).} % Título de la imágen
  \label{fig:figura01}                        % Anexar una etiqueta a la figura hace que sea fácil referenciarla después
\end{figure}

\subsection{Insertar tabla}

\begin{table}[ht!]                            % Insertar forzosamente aquí o en la parte superior
  \centering                                  % Centrado
  \begin{tabular}{lc}                         % Dos columnas, una alineado a la izquierda (l=left) y la otra centreada (c=center)
      \hline                                  % Línea horizontal
      \textbf{Columna 1} & \textbf{Columna 2} \\ % Cabecer de la tabla, ambas en negrillas
      \hline
      1,1 & 1,2 \\
      2,1 & 2,2 \\
      3,1 & 3,2 \\
      \hline
  \end{tabular}
  \caption{Tabla de prueba.}                  % Título de la tabla
  \label{table:tabla01}                       % Anexar una etiqueta a la tabla hace que sea fácil referenciarla después
\end{table}
\input{Capitulo2/MarcoTeorico.tex}
\input{Capitulo3/Metodologia.tex}
\input{Capitulo4/DisenoEImplementacion.tex}
\input{Capitulo5/Conclusiones.tex}

\backmatter     % Define lo que va después del contenido principal (Por ejemplo: Bibliografía o Anexos)

\bibliographystyle{apacite} % Listar las referencias bibliográficas en estilo APA
\bibliography{Bibliografia} % Nombre del archivo con extensión '.bib' donde están definidas las referencias bibliográficas

\addappheadtotoc              % Añade la entrada 'Anexos' a la tabla de contenidos
\begin{appendix}
  \appendixpage
  \begin{figure}
    \centering
    \includegraphics[scale=0.5]{cuadrado-rojo}\\
    Anexo 1: Un cuadrado rojo % Notar que no se utiliza el comando 'caption' ya que este agrega una entrada en la lista de figuras
  \end{figure}
\end{appendix}

\end{document}
